%%%%%%%%%%%%%%%%%%%%%%%%%%%%%%%%%%%%%%%%%
% Short Sectioned Assignment
% LaTeX Template
% Version 1.0 (5/5/12)
%
% This template has been downloaded from:
% http://www.LaTeXTemplates.com
%
% Original author:
% Frits Wenneker (http://www.howtotex.com)
%
% License:
% CC BY-NC-SA 3.0 (http://creativecommons.org/licenses/by-nc-sa/3.0/)
%
%%%%%%%%%%%%%%%%%%%%%%%%%%%%%%%%%%%%%%%%%

%----------------------------------------------------------------------------------------
%	PACKAGES AND OTHER DOCUMENT CONFIGURATIONS
%----------------------------------------------------------------------------------------

\documentclass[paper=a4, fontsize=11pt]{scrartcl} % A4 paper and 11pt font size

\usepackage[T1]{fontenc} % Use 8-bit encoding that has 256 glyphs
\usepackage{fourier} % Use the Adobe Utopia font for the document - comment this line to return to the LaTeX default
\usepackage[english]{babel} % English language/hyphenation
\usepackage{amsmath,amsfonts,amsthm} % Math packages

\usepackage{lipsum} % Used for inserting dummy 'Lorem ipsum' text into the template

\usepackage{sectsty} % Allows customizing section commands
\allsectionsfont{\centering \normalfont\scshape} % Make all sections centered, the default font and small caps

\usepackage{fancyhdr} % Custom headers and footers
\pagestyle{fancyplain} % Makes all pages in the document conform to the custom headers and footers
\fancyhead{} % No page header - if you want one, create it in the same way as the footers below
\fancyfoot[L]{} % Empty left footer
\fancyfoot[C]{} % Empty center footer
\fancyfoot[R]{\thepage} % Page numbering for right footer
\renewcommand{\headrulewidth}{0pt} % Remove header underlines
\renewcommand{\footrulewidth}{0pt} % Remove footer underlines
\setlength{\headheight}{13.6pt} % Customize the height of the header

%\numberwithin{equation}{section} % Number equations within sections (i.e. 1.1, 1.2, 2.1, 2.2 instead of 1, 2, 3, 4)
\numberwithin{figure}{section} % Number figures within sections (i.e. 1.1, 1.2, 2.1, 2.2 instead of 1, 2, 3, 4)
\numberwithin{table}{section} % Number tables within sections (i.e. 1.1, 1.2, 2.1, 2.2 instead of 1, 2, 3, 4)

\setlength\parindent{0pt} % Removes all indentation from paragraphs - comment this line for an assignment with lots of text

%----------------------------------------------------------------------------------------
%	TITLE SECTION
%----------------------------------------------------------------------------------------

\newcommand{\horrule}[1]{\rule{\linewidth}{#1}} % Create horizontal rule command with 1 argument of height

\title{	
\normalfont \normalsize 
\textsc{AST4320} \\ [25pt] % Your university, school and/or department name(s)
\horrule{0.5pt} \\[0.4cm] % Thin top horizontal rule
\huge Assignment 3 \\ % The assignment title
\horrule{2pt} \\[0.5cm] % Thick bottom horizontal rule
}

\author{Ina Kullmann} % Your name

\date{\normalsize\today} % Today's date or a custom date

\begin{document}

\maketitle % Print the title

%----------------------------------------------------------------------------------------
%	PROBLEM 1
%----------------------------------------------------------------------------------------

\section{The isothermal density profile}

The second order differential equation for $\rho(r)$ from the lecture notes is given by:

\begin{equation}
\frac{k_bT}{m_{DM}r^2} \frac{d}{dr}r^2 \frac{d}{dr} \ln \rho(r) = 4\pi G\rho(r)
\label{eq:DE}
\end{equation}

\textbf{Exercise:}  Show that $\rho(r) = \frac{A}{r^2}$, $A=\frac{k_bT}{2\pi Gm_{DM}}$ (the isothermal density profile) is a solution to the DE.\\

\textbf{Solution:} Rewriting the DE
\begin{align*}
\frac{k_bT}{2\pi gm_{DM}} \frac{1}{r^2} \frac{d}{dr}r^2 \frac{d}{dr} \ln \rho(r) = \frac{A}{r^2} \frac{d}{dr}r^2 \frac{d}{dr} \ln \rho(r) = 2\rho(r)
\end{align*}

Inserting the solution into the LHS of the DE:
\begin{align*}
\frac{A}{r^2} \frac{d}{dr}r^2 \frac{d}{dr} \ln \Bigg(\frac{A}{r^2} \Bigg) &=  \frac{A}{r^2} \frac{d}{dr}r^2 \frac{d}{dr} \Bigg( \ln A - \ln r^2 \Bigg) \\
&= -\frac{A}{r^2} \frac{d}{dr}r^2 \frac{d}{dr} \Bigg( 2 \ln r \Bigg) \\
&= -\frac{2A}{r^2} \frac{d}{dr}r^2 \cdot \frac{1}{r} \\
&= -\frac{2A}{r^2} \frac{d}{dr}r \\
&= -\frac{2A}{r^2} 
\end{align*}

Inserting the solution into the RHS of the DE:
\begin{align*}
2\rho(r) = \frac{2A}{r^2}
\end{align*}

\textbf{$\Rightarrow$ The LHS is equal to the RHS so that $\rho(r) = \frac{A}{r^2}$ is a solution to the DE.} \\

Gas in hydrostatic equilibrium with gravity is described by:
\begin{equation}
\frac{dP}{dr} = -\frac{GM(<r) \rho(r)}{r^2}
\label{eq:hyd_eq}
\end{equation}

\textbf{Exercise:} Show that an isothermal gas (obeying equation \ref{eq:hyd_eq}) settles into a similar state (as in equation \ref{eq:DE}).\\

\textbf{Solution:} If the gas is isothermal (it has the same temperature everywhere) then the pressure is given by
\begin{align*}
P = nk_bT_0 = \frac{\rho}{m_p}k_bT_0
\end{align*}

We use the definition
\begin{align*}
M(<r) = 4\pi \int_0^r x^2 \rho (x) dx
\end{align*}
and insert into equation \ref{eq:hyd_eq} to obtain the DE given in equation \ref{eq:DE}
\begin{align*}
\frac{d}{dr}\Bigg( \frac{k_bT}{m_p} \rho(r) \Bigg) &= -\frac{G\rho(r)}{r^2} \cdot 4\pi \int_0^r x^2 \rho (x) dx \\
\frac{k_bT}{m_p} \cdot \frac{r^2}{\rho(r)} \frac{d}{dr}  \rho(r)  &= - 4\pi G \int_0^r x^2 \rho (x) dx \\
\frac{d}{dr} \Bigg( \frac{k_bT}{m_p} \cdot \frac{r^2}{\rho(r)} \frac{d}{dr}  \rho(r) \Bigg)  &= \frac{d}{dr} \Bigg( - 4\pi G \int_0^r x^2 \rho (x) dx \Bigg) \\
\frac{k_bT}{m_p} \cdot \frac{d}{dr} \frac{r^2}{\rho(r)} \frac{d}{dr}  \rho(r) &= - 4\pi G  r^2 \rho (r)  \\
\frac{k_bT}{m_pr^2} \cdot \frac{d}{dr} r^2 \frac{d}{dr} \ln \rho(r) &= - 4\pi G  \rho (r)  \\
\end{align*}
where $\frac{d}{dr} \ln \rho(r) = \frac{1}{\rho(r)}\frac{d}{dr}\rho(r)$ is used in the last step. \textbf{This is the same DE as in equation \ref{eq:DE} which means that an isothermal gas in hydrostatic equilibrium with gravity settles into the same isothermal density profile as derived for the dark matter when it is assumed that the DM particles have the same velocity distrubution/temperature (?) everywhere in space. }


%----------------------------------------------------------------------------------------
%	PROBLEM 2
%----------------------------------------------------------------------------------------

\section{The Ly$\alpha$ forest}










%----------------------------------------------------------------------------------------

\end{document}