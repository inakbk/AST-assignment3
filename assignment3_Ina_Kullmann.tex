%%%%%%%%%%%%%%%%%%%%%%%%%%%%%%%%%%%%%%%%%
% Short Sectioned Assignment
% LaTeX Template
% Version 1.0 (5/5/12)
%
% This template has been downloaded from:
% http://www.LaTeXTemplates.com
%
% Original author:
% Frits Wenneker (http://www.howtotex.com)
%
% License:
% CC BY-NC-SA 3.0 (http://creativecommons.org/licenses/by-nc-sa/3.0/)
%
%%%%%%%%%%%%%%%%%%%%%%%%%%%%%%%%%%%%%%%%%

%----------------------------------------------------------------------------------------
%	PACKAGES AND OTHER DOCUMENT CONFIGURATIONS
%----------------------------------------------------------------------------------------

\documentclass[paper=a4, fontsize=11pt]{scrartcl} % A4 paper and 11pt font size

\usepackage[T1]{fontenc} % Use 8-bit encoding that has 256 glyphs
\usepackage{fourier} % Use the Adobe Utopia font for the document - comment this line to return to the LaTeX default
\usepackage[english]{babel} % English language/hyphenation
\usepackage{amsmath,amsfonts,amsthm} % Math packages

\usepackage{lipsum} % Used for inserting dummy 'Lorem ipsum' text into the template

\usepackage{sectsty} % Allows customizing section commands
\allsectionsfont{\centering \normalfont\scshape} % Make all sections centered, the default font and small caps

\usepackage{fancyhdr} % Custom headers and footers
\pagestyle{fancyplain} % Makes all pages in the document conform to the custom headers and footers
\fancyhead{} % No page header - if you want one, create it in the same way as the footers below
\fancyfoot[L]{} % Empty left footer
\fancyfoot[C]{} % Empty center footer
\fancyfoot[R]{\thepage} % Page numbering for right footer
\renewcommand{\headrulewidth}{0pt} % Remove header underlines
\renewcommand{\footrulewidth}{0pt} % Remove footer underlines
\setlength{\headheight}{13.6pt} % Customize the height of the header

%\numberwithin{equation}{section} % Number equations within sections (i.e. 1.1, 1.2, 2.1, 2.2 instead of 1, 2, 3, 4)
\numberwithin{figure}{section} % Number figures within sections (i.e. 1.1, 1.2, 2.1, 2.2 instead of 1, 2, 3, 4)
\numberwithin{table}{section} % Number tables within sections (i.e. 1.1, 1.2, 2.1, 2.2 instead of 1, 2, 3, 4)

\setlength\parindent{0pt} % Removes all indentation from paragraphs - comment this line for an assignment with lots of text

%----------------------------------------------------------------------------------------
%	TITLE SECTION
%----------------------------------------------------------------------------------------

\newcommand{\horrule}[1]{\rule{\linewidth}{#1}} % Create horizontal rule command with 1 argument of height

\title{	
\normalfont \normalsize 
\textsc{AST4320} \\ [25pt] % Your university, school and/or department name(s)
\horrule{0.5pt} \\[0.4cm] % Thin top horizontal rule
\huge Assignment 3 \\ % The assignment title
\horrule{2pt} \\[0.5cm] % Thick bottom horizontal rule
}

\author{Ina Kullmann} % Your name

\date{\normalsize\today} % Today's date or a custom date

\begin{document}

\maketitle % Print the title

%----------------------------------------------------------------------------------------
%	PROBLEM 1
%----------------------------------------------------------------------------------------

\section{The isothermal density profile}

The second order differential equation for the density of the dark matter, $\rho(r)$ is given by:

\begin{equation}
-\frac{k_bT}{m_{DM}r^2} \frac{d}{dr}r^2 \frac{d}{dr} \ln \rho(r) = 4\pi G\rho(r)
\label{eq:DE}
\end{equation}

\textbf{Exercise:}  Show that $\rho(r) = \frac{A}{r^2}$, $A=\frac{k_bT}{2\pi Gm_{DM}}$ (the isothermal density profile) is a solution to the DE (that the DM settles into an isothermal density profile).\\

\textbf{Solution:} Rewriting the DE
\begin{align*}
-\frac{k_bT}{2\pi gm_{DM}} \frac{1}{r^2} \frac{d}{dr}r^2 \frac{d}{dr} \ln \rho(r) = -\frac{A}{r^2} \frac{d}{dr}r^2 \frac{d}{dr} \ln \rho(r) = 2\rho(r)
\end{align*}

Inserting the solution into the LHS of the DE:
\begin{align*}
-\frac{A}{r^2} \frac{d}{dr}r^2 \frac{d}{dr} \ln \Bigg(\frac{A}{r^2} \Bigg) &=  -\frac{A}{r^2} \frac{d}{dr}r^2 \frac{d}{dr} \Bigg( \ln A - \ln r^2 \Bigg) \\
&= \frac{A}{r^2} \frac{d}{dr}r^2 \frac{d}{dr} \Bigg( 2 \ln r \Bigg) \\
&= \frac{2A}{r^2} \frac{d}{dr}r^2 \cdot \frac{1}{r} \\
&= \frac{2A}{r^2} \frac{d}{dr}r \\
&= \frac{2A}{r^2} 
\end{align*}

Inserting the solution into the RHS of the DE:
\begin{align*}
2\rho(r) = \frac{2A}{r^2}
\end{align*}

\textbf{$\Rightarrow$ The LHS is equal to the RHS so that $\rho(r) = \frac{A}{r^2}$ is a solution to the DE.} \\

Gas (baryons) in hydrostatic equilibrium with gravity is described by:
\begin{equation}
\frac{dP}{dr} = -\frac{GM(<r) \rho_b(r)}{r^2}
\label{eq:hyd_eq}
\end{equation}
where the pressure $P$ is given by the baryons only, since the DM is pressureless. The right hand side is describing the gravitational pull from $M(<r)$, both the baryons and the DM, on the baryons $\rho_b$. \\

\textbf{Exercise:} Show that an isothermal gas (obeying equation \ref{eq:hyd_eq}) settles into a similar state (same density profile as the DM).\\

\textbf{Solution:} We want to show that the baryons settle into the same density profile as the DM, we want to show that
\begin{align*}
\rho_b(r) = \frac{B}{r^2}
\end{align*}
gives a solution to equation \ref{eq:hyd_eq}, where $B$ is a constant. If the gas is isothermal (it has the same temperature $T_0$ everywhere) then the pressure of the baryons are given by
\begin{align*}
P = nk_bT_0 = \frac{\rho_b}{m_p}k_bT_0 = \frac{k_bT_0B}{m_pr^2}
\end{align*}
giving the LHS:

\begin{align*}
\frac{d}{dr} \frac{k_bT_0B}{m_pr^2} = \frac{k_bT_0B}{m_p} \cdot -2r^{-3} = -\frac{2k_bT_0B}{m_pr^3}  
\end{align*}

The total mass enclosed in the sphere is given by the integral
\begin{align*}
M(<r) &= 4\pi \int_0^r x^2 \rho_{tot} (x) dx = 4\pi \int_0^r x^2 \bigg(\rho_b(x) + \rho_{DM}(x)\bigg) dx \\
&= 4\pi \int_0^r x^2 \Bigg(\frac{B}{x^2} + \frac{A}{x^2}\Bigg) dx = 4\pi (A+B)r
\end{align*}
so that the RHS becomes:
\begin{align*}
-\frac{GM(<r)\rho_b(r)}{r^2} = -\frac{G\cdot  4\pi (A+B)r \cdot \frac{B}{r^2}}{r^2} = -\frac{4\pi G(A+B)B}{r^3}
\end{align*}
setting the LHS equal to the RHS:
\begin{align*}
-\frac{2k_bT_0B}{m_pr^3} &= -\frac{4\pi G(A+B)B}{r^3} \\
\Rightarrow \frac{k_bT_0}{m_p} &= 2\pi G(A+B) \\
\Rightarrow  B &= \frac{k_bT_0}{2\pi G m_p} - A 
\end{align*}
\textbf{The baryons settle into the isothermal density profile if $B$ has the value given above, and that the DM also follows an isothermal density profile. }

%----------------------------------------------------------------------------------------
%	PROBLEM 2
%----------------------------------------------------------------------------------------

\section{The Ly$\alpha$ forest}

\textbf{Exercise:} Compute the Jeans length in the ionized IGM as a function of redshift. Assume $T_{IGM} = 10^4$K. What value of $k$ does this correspond to?

\textbf{Solution:} The Jeans mass and length is given by:
\begin{align*}
M_J &= \frac{4\pi}{3} \Bigg( \frac{\lambda_J}{2} \Bigg)^3 \rho_b = \frac{\pi^{5/2}}{6G^{3/2}\rho_b^{1/2}} \Bigg( \frac{k_bT}{\mu m_p} \Bigg)^{3/2} \\
\Bigg( \frac{\lambda_J}{2} \Bigg)^3 &= \frac{3}{4\pi}\frac{1}{\rho_b} \cdot \frac{\pi^{5/2}}{6G^{3/2}\rho_b^{1/2}} \Bigg( \frac{k_bT}{\mu m_p} \Bigg)^{3/2} \\
&= \frac{\pi^{3/2}}{8G^{3/2}\rho_b^{3/2}} \Bigg( \frac{k_bT}{\mu m_p} \Bigg)^{3/2} = \frac{1}{8} \Bigg(\frac{\pi}{G\rho_b}  \frac{k_bT}{\mu m_p} \Bigg)^{3/2}  \\
\lambda_J &= 2\cdot \Bigg( \frac{1}{8} \Bigg)^{1/3} \Bigg( \frac{\pi k_bT}{G\rho_b\mu m_p} \Bigg)^{1/2}
\end{align*}

from the lecture notes we know that
\begin{align*}
\bar{n}_H \approx 2 \cdot 10^{-7} (1 + z)^3 \mathrm{cm}^{-3}
\end{align*}
and that
\begin{align*}
\bar{n}_H  = f_H\frac{\rho_p}{m_p} = 0.76 \frac{\rho_b}{m_p} \Rightarrow \rho_b = \frac{m_p}{0.76}\bar{n}_H 
\end{align*}
inserting the numbers gives the Jeans length:










%----------------------------------------------------------------------------------------

\end{document}